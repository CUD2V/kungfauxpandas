\documentclass{article}

% if you need to pass options to natbib, use, e.g.:
% \PassOptionsToPackage{numbers, compress}{natbib}
% before loading nips_2018

% ready for submission
%\usepackage{nips_2018}

% to compile a preprint version, e.g., for submission to arXiv, add
% add the [preprint] option:
\usepackage[preprint]{nips_2018}

% to compile a camera-ready version, add the [final] option, e.g.:
% \usepackage[final]{nips_2018}

% to avoid loading the natbib package, add option nonatbib:
% \usepackage[nonatbib]{nips_2018}

\usepackage[utf8]{inputenc} % allow utf-8 input
\usepackage[T1]{fontenc}    % use 8-bit T1 fonts
\usepackage{hyperref}       % hyperlinks
\usepackage{url}            % simple URL typesetting
\usepackage{booktabs}       % professional-quality tables
\usepackage{amsfonts}       % blackboard math symbols
\usepackage{nicefrac}       % compact symbols for 1/2, etc.
\usepackage{microtype}      % microtypography
\usepackage[super]{nth}     % JK: for 1st-ing and 2nd-ing

\usepackage{graphicx}       % added by seth - image support
\graphicspath{ {./images/} }
\usepackage{soul}           % added by seth - remove before submission, provides highlighting support
\usepackage{wrapfig}        % added by seth for placing images in-line with text

\usepackage{hyperref}
\hypersetup{
    colorlinks=true,
    linkcolor=blue,
    filecolor=magenta,      
    urlcolor=blue,
}
 
\urlstyle{same}
\title{%
  Kung Faux Pandas \\
  \large Simplifying Privacy Protection
  }

\author{
  James King, Seth Russell, Tellen D. Bennett, Debashis Ghosh\\
  Data Science to Patient Value (D2V)\\
  School of Medicine\\
  University of Colorado Anschutz Medical Campus\\
  Aurora, CO\\
  \texttt{\{james.king, seth.russell, tell.bennett, debashis.ghosh\}@ucdenver.edu}
  }

\date{May 2018}

\begin{document}

% Advice from Tell:  articulate the problem with evidence (i.e. papers)
%   Enable reproducibility - patient value, make process of replication easier
%   Time saving
%   legal risk (carrying data in laptops, etc)
%   Multiple comparison?
%   Educational data sets
%  Definition of :reproducibility 
%       reproducibilty - everything same - code, data, result
%       replicability - code same, new data, same result find Peng & Leek reference (also look at blog)
%   unit testing...
%   Proof of privacy guarantees?

\maketitle

\begin{abstract}
A description of an end-to-end system for easily generating faux-data which is statistically similar to given real data but does not contain sensitive personal information.  This system, which is publicly available \footnote{\url{https://github.com/CUD2V/kungfauxpandas}}, enables data to be distributed for reproducibility testing and other purposes while in full compliance with HIPAA and GDPR.
\end{abstract}


\section{Introduction}

The ability of independent investigators to verify results is perhaps \emph{the} most fundamental feature that separates science from other forms intellectual inquiry.  Unfortunately, the mechanisms of communicating scientific results have changed little since the age of enlightenment and are inadequate enabling reproducibility for some modern scientific endeavors such as machine learning.  Perhaps the most important impediment to reproducibility is the need to share data.  This has long been difficult in health care fields, but will presently become difficult for just about everyone.

There is a large body of published literature detailing methods for overcoming this difficulty via ``anonymizing'' and ``synthesising'' data \cite{patki_synthetic_2016, choi_generating_2017}, however at present a mechanism to routinely \emph{use} these techniques has not been forthcoming.

% explain faux here? faux data - similar to faux fur or faux leather - ethically more appealing, less expensive to produce or use
% faux data is a broader term that encompasses any data privacy protection method - synthesis, anonymization, grouping, perturbation, etc.
This paper details an open source software library called \emph{Kung Faux Pandas (KFP)} which allows for the modular use of data security methods into the popular Python Pandas data science library. For those who prefer other tools, KFP also provides Structured Query Language (SQL) interface which enables users to arbitrarily query any data within a database, returning to the user a data set which either does not contain any personal information (synthesized) or has been anonymized by standard methods.

\section{Background}

In the United States, there's been a long-standing regulation described in the Health Insurance Portability and Accountability Act (HIPAA), which mandates strong security on data related to health care and is backed by significant fines and possible criminal charges \cite{hippaviol}.  Thus, even institutions which share data face a steep institutional risk. This makes it virtually impossible to enable independent verification of machine learning (ML) models which often require enormous amounts of training data.

Perhaps more ominously for researchers, on May 25 , 2018, the European Union's (EU) General Data Protection Regulation (GDPR) has come into full effect \cite{gdpr}, establishing a new paradigm in data ``ownership'' for that jurisdiction.   In short, this set of rules defines the ``owners'' of data to be people from whom it was collected and obliges anyone possessing  that data to  honor the owners' preferences about how the data is kept and used, including a requirement to deletion of any data at the request of its owner at any time in the future.

These rules are ultimately good for citizens, but they create a pickle for scientists studying humans as it's very difficult to follow normal processes in scientific rigor while complying with privacy protection rules.

One obvious way around this pickle is to separate data and analysis in such a way that the analysis can be performed on data without the \emph{analyst} having access to the data.   This is fairly straight-forward to implement.   Essentially, the analyst, with only a description of the sensitive data, composes analytical code which is handed off to the data custodian.  The custodian then runs it and returns the results to the researcher.

This has been technologically possible for some time, but it's awkward in practice because activities such as data cleaning, exploration, and plotting are interactive and iterative processes which would become impractical with a mediator.

Kung Faux Pandas sidesteps this problem by generating \emph{faux-data} that is ``statistically similar'' to the real data but does not itself contain any sensitive data.

This faux-data can be prodded, poked, plotted, and posted for the world to see, enabling analysts to develop their code using whatever means they prefer.   When the analytical code is ready, it can then be sent to the data custodian to run on the real data.


\section{System Details}

% some text about goal to make our research reproducible/replicable - provide source code, pip/conda environment files, docker container

Many methods of generating faux-data exist in the literature, many of which have software implementations freely available.   KFP provides a standard mechanism for ``wrapping'' any method into a ``plug-in'' which integrates the method with the Pandas data frame model.  Three of these plug-ins are provided with full documentation for creating other plug-ins.

\subsection{System Structure}
KFP provides a class

Describe KFP's structure, including the ``plugin'' mechanism.

Describe the SQL interface

The ideal environment for a machine learning developer to develop, test, or validate a model is to have unfettered access to all data in the problem domain they are working in. However, as discussed previously, legal limitations and data protection processes are slow. Figure~\ref{fig:architecture} depicts a basic architecture of an alternative to direct data access: data access through an intermediary that can synthesize data on-demand.

\begin{figure}[ht]
  \centering
  \includegraphics[width=\textwidth]{prototype_architecture}
  \caption{System Architecture.}
  \label{fig:architecture}
\end{figure}

Figure~\ref{fig:synthesis_process} provides a detailed look at the automated faux data request/response process. All queries submitted to the system are first pre-processed to remove clauses such as "ORDER BY" which are made ineffective by the synthesis process. All queries are logged for later analysis and results cached to optionally improve synthesis performance for repeated queries. Next the processed query is executed against the protected data set and results are passed into the modular synthesis process. While the current implementation only utilizes a couple of synthesis methods, machine learning developers or data warehouse owners can insert their own custom method or modify existing methods. Previously removed "ORDER BY" clauses are re-applied to the faux data before results are returned to the user. Finally, although not implemented in this version of KFP, synthetic results could be held back until reviewed and approved for release by the data owner.


\begin{figure}[ht]
  \centering
  \includegraphics[width=120mm]{data_synthesis_process}
  \caption{Data synthesis process.}
  \label{fig:synthesis_process}
\end{figure}


% figure out how to adjust padding
\begin{wrapfigure}{r}{0.5\textwidth}%optionally place [15] after {wrapfigure} to control how many lines to wrap around figure
  \centering
  \includegraphics[width=0.5\textwidth]{ui_screenshot3}
  \caption{User Interface.}
  \label{fig:ui}
\end{wrapfigure}
KFP provides multiple methods by which a machine learning developer can access on-demand faux data. As shown in Figure~\ref{fig:ui}, a single page web application (SPA) allows for users to directly enter SQL queries, submit them to the connected database, and receive faux data according to the selected generation method. Data can either be viewed directly in browser or downloaded in csv format for use in a machine learning process. Alternatively, an http REST service (utilized by the SPA) allows for cross language compatible requesting and receiving of faux data, again via SQL query. Lastly, for Python based software or languages with an interface to Python, the kungfauxpandas.py file and associated classes can be imported and called natively.

\subsection{Included Methods}

Some text about Table~\ref{included-methods} including why we built the KDE method, why DataSynthesizer, and why SGF% Not sure where this table should go....

\begin{table}
  \caption{Kung Faux Pandas data synthesis methods}
  \label{included-methods}
  \centering
  \begin{tabular}{p{12em} l p{14em}}
    \toprule
    Method                                 & Source                    & Notes \\
    \midrule
    Kernel Density Estimator               & scipy.stats.gaussian\_kde & Inter-related ordinal and ratio data\\
    DataSynthesizer                        & Ping et al.               & Bayesian modeling of column relationships\\
    Synthetic dataset Generation Framework & Bindschaedler et al.      & "Plausible Deniability" probabilistic modeling\\
    \bottomrule
  \end{tabular}
\end{table}

\section{Discussion}

Limitations, areas for further research - convert this to paragraph format:

\begin{itemize}
    \item Python module
    \item Currently don't impose any kind of result set minimum size. Query pre-processor and synthesis process could optionally require a minimum source data size to improve privacy protection...
    \item Use in production
\end{itemize}

%* Minimum group/result set size depends on several factors such as re-identification risk, number of attributes available (see picture at end)
%* Individual states have adopted minimum group size reporting rules, with the minimum number of students ranging from 5 to 30 and a modal category of 10 (used by 39 states) https://nces.ed.gov/pubs2011/2011603.pdf
%* 62 https://nces.ed.gov/nationsreportcard/tdw/analysis/summary_rules_minimum.aspx
%* 30; describes techniques to determine size - https://nces.ed.gov/pubs2017/2017147.pdf

\section{Conclusion}

Basically need to put abstract here...


\bibliographystyle{unsrt}

\bibliography{bib}

\end{document}