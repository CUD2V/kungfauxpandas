\documentclass{article}


%\usepackage{nips_2017}
\usepackage[final]{nips_2017}
% to compile a camera-ready version, add the [final] option, e.g.:
% \usepackage[final]{nips_2017}

\usepackage[utf8]{inputenc} % allow utf-8 input
\usepackage[T1]{fontenc}    % use 8-bit T1 fonts
\usepackage{hyperref}       % hyperlinks
\usepackage{url}            % simple URL typesetting
\usepackage{booktabs}       % professional-quality tables
\usepackage{amsfonts}       % blackboard math symbols
\usepackage{nicefrac}       % compact symbols for 1/2, etc.
\usepackage{microtype}      % microtypography

\usepackage{hyperref}
\hypersetup{
    colorlinks=true,
    linkcolor=blue,
    filecolor=magenta,      
    urlcolor=blue,
}
 
\urlstyle{same}
\title{%
  Kung Faux Pandas \\
  \large A tool for enabling data sharing in health care}

%\author{Seth Russel  \& James King}
\author{
  James King \& Seth Russell\\
  Data Science to Patient Value (D2V)\\
  School of Medicine \\
  University of Colorado Anschutz Medical Campus\\
%  Aurora, CO\\
  \texttt{seth.russell@ucdenver.edu} \\
  \texttt{james.king@ucdenver.edu} \\ 
  }

\date{May 2018}

\begin{document}

% Advice from Tell:  articulate the problem with evidence (i.e. papers)
%   Enable reproducibility - patient value, make process of replication easier
%   Time saving
%   legal risk (carrying data in laptops, etc)
%   Multiple comparison?
%   Educational data sets
%  Definition of :reproducibility 
%       reproducibilty - everything same - code, data, result
%       replicability - code same, new data, same result find Peng & Leek reference (also look at blog)
%   unit testing...

\maketitle

\begin{abstract}
A description of an end-to-end system for easily generating synthetic data which contains no Personal Health Information (PHI), allowing example data to be distributed for reproducibility testing and other purposes.  

\end{abstract}



\section{Introduction}

Reproducing machine learning results in the field of health care is challenging. Many published research articles do not make code nor data available. Journals could require the inclusion of code in submissions, however most health care data is very strictly protected by law \cite{hippapro}.  

Various approaches to building data sets which simultaneously have scientific utility and comply with privacy laws.  Broadly speaking, these methods fall into two categories: anonymization and synthesis.  Anonymization methods attempt to keep as much of the "real" data as possible, whilst ensuring that person-specific data 

\section{Background}

 Over the past 18 years, there has been several significant papers clarifying the definition and importance of reproducibility along with the closely related term replicability. Leek and Peng have defined reproducibility as the "ability to recompute data analytic results given an observed dataset and knowledge of the data analysis pipeline," and replicability as "the chance that an independent experiment targeting the same scientific question will produce a consistent result" \cite{leek_opinion_2015}; others such as Drummond use the terms in a reverse fashion \cite{drummond_replicability_2009}. Despite the semantic disagreement, both groups agree that an independent experiment with confirmatory results is the strongest support of any experiment. One early influential paper on the topic of reproducibility in scientific computing details key factors in reproducbility as having: data, input parameters, documenation, software code, and an environment capable of running the provided software code; items that cannot easily nor clearly be communicated by paper \cite{schwab_making_2000}. Reproducibility should be treated as a minimium required standard that all published research should meet \cite{peng_reproducible_2006}. Specifically in the context of machine learning, this means including details to a level someone else could create the same environment including hardware configuration and run times, data used for all experiments, source code, documentation on data and how to run/configure software, and tests that verify the software runs correctly. The last item is important but often overlooked - unit tests can show how a researcher validated their code was giving expected results which can often lead to insights not normally communicated in a results section of a journal article. Once a result can be reproduced, new researchers can then build upon the methods, gather new data for testing/validation, or discover alternative methods to replicate a result.
 

\section{Appendix: Legal References}
In the U.S., the primary law affecting health data privacy is the Health Insurance Portability and Accountability Act (\href{https://www.hhs.gov/hipaa/for-professionals/index.html}{HIPAA}). 


Additionally, organizations desiring to perform biomedical and social, behavioral, and educational research are subject to Common Rule (\href{https://www.hhs.gov/ohrp/regulations-and-policy/regulations/45-cfr-46/index.html#subparta}{45 CFR 46, Subpart A}) and/or the U.S. Food and Drug Administration’s regulations (\href{https://www.ecfr.gov/cgi-bin/text-idx?SID=faa4b2b2900a70fbcac4a773c9da0f0f&mc=true&node=pt21.1.50&rgn=div5}{21 CFR 50} and \href{https://www.ecfr.gov/cgi-bin/text-idx?SID=faa4b2b2900a70fbcac4a773c9da0f0f&mc=true&node=pt21.1.56&rgn=div5}{56}). Furthermore, the Common Rule established Institutional Review Boards which are responsible and the institutional level of safe guarding human subjects in any research. 


\bibliographystyle{unsrt}
\bibliography{bib}

\end{document}