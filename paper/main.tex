\documentclass{article}

%\usepackage{nips_2017}
\usepackage[final]{nips_2017}
% to compile a camera-ready version, add the [final] option, e.g.:
% \usepackage[final]{nips_2017}

\usepackage[utf8]{inputenc} % allow utf-8 input
\usepackage[T1]{fontenc}    % use 8-bit T1 fonts
\usepackage{hyperref}       % hyperlinks
\usepackage{url}            % simple URL typesetting
\usepackage{booktabs}       % professional-quality tables
\usepackage{amsfonts}       % blackboard math symbols
\usepackage{nicefrac}       % compact symbols for 1/2, etc.
\usepackage{microtype}      % microtypography


\title{%
  Kung Faux Pandas \\
  \large Alternative Facts for 
    Privacy Preservation}

%\author{James King \& Seth Russel}
\author{
  James King \& Seth Russell\\
  Data Science to Patient Value (D2V) \\
  School of Medicine \\
  University of Colorado Anschutz Medical Campus\\
  \texttt{james.king@ucdenver.edu} \\ 
  \texttt{seth.russell@ucdenver.edu} \\
  }

\date{May 2018}

\begin{document}

% Tell:  articulate the problem with evidence (i.e. papers)
%   Enable reproducibility - patient value, make process of replication easier
%   Time saving
%   legal risk (carrying data in laptops, etc)
%   Multiple comparison?
%   Educational data sets


\maketitle

\begin{abstract}

\end{abstract}

\section{Introduction}

The development and widespread use of literate programming tools such as RMarkdown and Jupyter Notebooks used in conjunction with public code archives like GitLab have enabled a giant leap forward in reproducible research.  It is now possible to write documents which contain source code, visualizations, analytical results, and narrative writing in a single package which can distributed to any interested person for essentially zero cost.

While this pattern (workflow? combination?) makes it easy for readers to check source code for errors, it only enables reproducible research if the 

test of autogit script

\bibliographystyle{unsrt}
\bibliography{bib}

\end{document}