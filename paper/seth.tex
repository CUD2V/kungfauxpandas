"present papers from the coding perspective so that reproducibility and replication of results in the Machine Learning community becomes easier"

* Reproducing Machine learning requires access to source data
* Source data is hard to get
* In medical field, harder to get than most fields – IRBs, HIPAA regulation, etc.
* Various methods used to synthesize or de-identify

A wide array of methods and techniques have been developed for generating synthetic data sas  data \cite{walonoski_synthea:_2018}

Need something similar to this - part of our motivation for our work:
"in the U.S., the Privacy Rule of the Health Insurance Portability and Accountability Act (HIPAA)  outlines two policies for protecting anonymity, namely Safe Harbor, and Expert Determination. The first of these policies enumerates eighteen direct identifiers that must be removed from data, prior to their dissemination, while, according to the Expert Determination policy, an expert needs to certify that the data to be disseminated pose a low privacy risk before the data can be shared with external parties." (from Publishing data from electronic health records while preserving privacy: A survey of algorithms)

What reference to use for HIPAA? \cite{hippapro}

Need some information (and reference(s)) about IRBs and human subjects research. 

Re-identification attacks - should we include brief description about this? Some references: 
* El Emam K, Jonker E, Arbuckle L, Malin B. A systematic review of re-identification attacks on health data. PLoS ONE 2011;6(12) [e28071, 12].
* Sweeney L. k-anonymity: a model for protecting privacy. Ijufks 2002;10:557–70.

Reproducing machine learning results in the field of health care is challenging. Many published research articles do not make code available nor data available. The first issue can be addressed by journal requirements, but the second is much more difficult. In the domain of health care in the U.S., the primary law affecting health data privacy is the Health Insurance Portability and Accountability Act. Additionally, organizations desiring to perform biomedical and social, behavioral, and educational research are subject to Common Rule (45 CFR 46, Subpart A) and/or the U.S. Food and Drug Administration’s regulations (21 CFR 50 and 56). Furthermore, the Common Rule established Institutional Review Boards which are responsible and the institutional level of safe guarding human subjects in any research. 

